% Test if we're on the second pass by seeing if the BBL already exists
\newif\ifsecond
\newread\r
\openin\r=bibtex_multiple_aux_files.bbl
\ifeof\r
\message{first pass}
\secondfalse
\else
\message{second pass}
\secondtrue
\closein\r
\fi

% Now create first .aux
\newwrite\w
\immediate\openout\w=bibtex_multiple_aux_files.aux\relax
\immediate\write\w{\string\bibdata{refs}}
\immediate\write\w{\string\citation{refA}}
\immediate\write\w{\string\bibstyle{catchkey}}
\immediate\closeout\w

% Second .aux
\immediate\openout\w=secondary.aux\relax
\immediate\write\w{\string\bibdata{refs}}
\immediate\write\w{\string\citation{refB}}
\immediate\write\w{\string\bibstyle{catchkey}}
\immediate\closeout\w

hello

% catchkey.bst emits a command `\saw$KEY` for each key it sees. So
% we can check that we got keys from both files like so:

\newif\ifrefA
\refAfalse

\newif\ifrefB
\refBfalse

\ifsecond
  \let\sawrefA=\refAtrue
  \let\sawrefB=\refBtrue
  % Test if we're on the second pass by seeing if the BBL already exists
\newif\ifsecond
\newread\r
\openin\r=bibtex_multiple_aux_files.bbl
\ifeof\r
\message{first pass}
\secondfalse
\else
\message{second pass}
\secondtrue
\closein\r
\fi

% Now create first .aux
\newwrite\w
\immediate\openout\w=bibtex_multiple_aux_files.aux\relax
\immediate\write\w{\string\bibdata{refs}}
\immediate\write\w{\string\citation{refA}}
\immediate\write\w{\string\bibstyle{catchkey}}
\immediate\closeout\w

% Second .aux
\immediate\openout\w=secondary.aux\relax
\immediate\write\w{\string\bibdata{refs}}
\immediate\write\w{\string\citation{refB}}
\immediate\write\w{\string\bibstyle{catchkey}}
\immediate\closeout\w

hello

% catchkey.bst emits a command `\saw$KEY` for each key it sees. So
% we can check that we got keys from both files like so:

\newif\ifrefA
\refAfalse

\newif\ifrefB
\refBfalse

\ifsecond
  \let\sawrefA=\refAtrue
  \let\sawrefB=\refBtrue
  % Test if we're on the second pass by seeing if the BBL already exists
\newif\ifsecond
\newread\r
\openin\r=bibtex_multiple_aux_files.bbl
\ifeof\r
\message{first pass}
\secondfalse
\else
\message{second pass}
\secondtrue
\closein\r
\fi

% Now create first .aux
\newwrite\w
\immediate\openout\w=bibtex_multiple_aux_files.aux\relax
\immediate\write\w{\string\bibdata{refs}}
\immediate\write\w{\string\citation{refA}}
\immediate\write\w{\string\bibstyle{catchkey}}
\immediate\closeout\w

% Second .aux
\immediate\openout\w=secondary.aux\relax
\immediate\write\w{\string\bibdata{refs}}
\immediate\write\w{\string\citation{refB}}
\immediate\write\w{\string\bibstyle{catchkey}}
\immediate\closeout\w

hello

% catchkey.bst emits a command `\saw$KEY` for each key it sees. So
% we can check that we got keys from both files like so:

\newif\ifrefA
\refAfalse

\newif\ifrefB
\refBfalse

\ifsecond
  \let\sawrefA=\refAtrue
  \let\sawrefB=\refBtrue
  % Test if we're on the second pass by seeing if the BBL already exists
\newif\ifsecond
\newread\r
\openin\r=bibtex_multiple_aux_files.bbl
\ifeof\r
\message{first pass}
\secondfalse
\else
\message{second pass}
\secondtrue
\closein\r
\fi

% Now create first .aux
\newwrite\w
\immediate\openout\w=bibtex_multiple_aux_files.aux\relax
\immediate\write\w{\string\bibdata{refs}}
\immediate\write\w{\string\citation{refA}}
\immediate\write\w{\string\bibstyle{catchkey}}
\immediate\closeout\w

% Second .aux
\immediate\openout\w=secondary.aux\relax
\immediate\write\w{\string\bibdata{refs}}
\immediate\write\w{\string\citation{refB}}
\immediate\write\w{\string\bibstyle{catchkey}}
\immediate\closeout\w

hello

% catchkey.bst emits a command `\saw$KEY` for each key it sees. So
% we can check that we got keys from both files like so:

\newif\ifrefA
\refAfalse

\newif\ifrefB
\refBfalse

\ifsecond
  \let\sawrefA=\refAtrue
  \let\sawrefB=\refBtrue
  \input{bibtex_multiple_aux_files.bbl}
  \input{secondary.bbl}
  \ifrefA\else
    \didnotseerefA
  \fi
  \ifrefB\else
    \didnotseerefB
  \fi
\fi

\bye

  \input{secondary.bbl}
  \ifrefA\else
    \didnotseerefA
  \fi
  \ifrefB\else
    \didnotseerefB
  \fi
\fi

\bye

  \input{secondary.bbl}
  \ifrefA\else
    \didnotseerefA
  \fi
  \ifrefB\else
    \didnotseerefB
  \fi
\fi

\bye

  \input{secondary.bbl}
  \ifrefA\else
    \didnotseerefA
  \fi
  \ifrefB\else
    \didnotseerefB
  \fi
\fi

\bye
